\section{approach}
\label{sec:approach}

\subsection {\bf Testbed Setup}
We plan to run several kinds of workload on WAIL machines to collect traces. Currently we could use up to 12 machines with 3-tier tree topology. We could also change the network topology and the oversubscription factor. 

\subsection {\bf Measurement}
We plan to run a couple of representative Hadoop workloads on HDFS and collect network/storage traces.

We try to answer this question -- how are storage and network behaviors related with each other? For example,
1) Is the bursty network traffic correlated with bursty storage activity at the same time? 
2) How does storage background traffic influence the normal data transfer? How does the normal traffic relate to storage activity?
3) How does the network topology and workload type affect the above network/storage interaction?

\begin{itemize}
	\item {\bf Metrics}
We need to do some instrumentation in Hadoop and HDFS for collecting traces.

Overall, we care about the job completion time. 

For HDFS/storage, we plan to measure the following metrics:
1) Disk-level behavior (bandwidth, IOPS) and HDFS-level behavior (throughput, I/O request rate)
2) Bandwidth/throughput v.s. block size

For network, we plan to measure the following metrics:
1) Flow duration, distribution
2) Flow rate, distribution

	\item {\bf Workloads}
We consider some representative Hadoop workloads:
1) K-means: CPU intensive
2) Terasort: Moderate CPU, Moderate I/O
3) Grep: Moderate CPU and small I/O
	
	\item {\bf Scenarios}
1) Single workload
2) Mixed workload

\end{itemize}

\subsection{\bf Initial Simulations}
After analyzing traces, we expect to come up with some heuristics and conduct initial simulations to demonstrate the benefits of network-aware storage. Some of the heuristics we have in mind now are:
1) When reading blocks, if the data locality cannot be met, rather than randomly schedule a map task with the nearest input replica, we could choose a map task with the replica which it could fetch most quickly. When writing blocks, rather than randomly place a replica on a node in remote rack, we may also take how congested the link is into consideration when choosing the replica location/path to the destination.
2) Put off HDFS block writes. We may put off HDFS writes of some replicas of a block when some links are heavy loaded.

