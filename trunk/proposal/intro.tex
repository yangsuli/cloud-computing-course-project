\section{Introduction}
\label{section:intro}

It has becoming increasingly popular to talk about "cloud computing", and Infrastructure as a Service (SaaS) has been offered by several giant providers, like Amazon, Microsoft and Google. A lot of computation has been migrated to the public cloud, and a variety of application are now been hosted in cloud. Different cloud offerings differ in performance characters, pricing model and costs. A throughout and in-depth comparison across different cloud offerings would guide people choose appropriate platform for their computation and minimize costs; it might also offer application designers some insights on how to engineer their applications to better leverage these services. However, few studies have been done in this space. This paper aims at providing some preliminary performance comparison across major cloud offerings, and some insights on how applications would perform on those platforms and what design considerations should be taken into account when leveraging these services. 

Following the methodology used by one prior research on comparing public cloud providers, we roughly divide the functionalities SaaS offers into three categories: elastic compute cluster, persistent storage and network (both intra-cloud and wide-area network. We used well-established benchmarks to measure each functionality on both Amazon EC2/EBS and Microsoft, and reported the results. We also strived to compare these results in a meaningful way, and offer good recommendations for application designers planning to use cloud services. 

The rest of this paper is organized as the following: in section 3 we measured the performance of public cloud as a computing cluster; in section 4 we measured and compared the performance of the persistent storage offered by public cloud; in section 5 we measured the intra-cloud and wide-area network performance; finally we concluded with our take-aways of these performance measurement numbers in section 6.
